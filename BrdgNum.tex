\documentclass[11pt, oneside]{article}   
\usepackage{geometry}                		
\geometry{letterpaper}                   		   		
\usepackage{graphicx}	
\usepackage[utf8]{inputenc}
\usepackage{mathtools}
\usepackage[english]{babel}			
\usepackage[usenames, dvipsnames]{color}								
\usepackage{amssymb}
\usepackage{amsmath, amsthm}
\newtheorem*{problem*}{Problem}
\newtheorem*{solution*}{Solution}
\newtheorem*{lemma*}{Lemma}
\newtheorem*{definition*}{Definition}

\title{Annotations to ``Additivity of Bridge Numbers of Knots''}
\author{Adam Howard}
\date{\today}							

\begin{document}
\maketitle



\section*{The Result and the Setup}
In Jennifer Schultens's paper ``Additivity of Bridge Numbers of knots'' the author gives a new proof that if $K = J \# L$ is the connect sum of knots, then $b(K) = b(J) + b(L) - 1.$ The connect sum of knots can be realized using the satellite construction and Schultens also proves that if $K$ is a satellite knot with companion $J$ and whose pattern has index $k$, then $b(K) \geq k\cdot b(J).$ These results were originally proven by Horst Schubert. \newline \newline The setting for this paper will be $K \subset V  \subset S^{3},$ where $K$ is a knot and $V$ is a knotted solid torus. We will also have $h: S^{3} \rightarrow \mathbb{R}^{3}$ a Morse function with exactly two critical points, which guarantees that $h$ induces a foliation of $S^{3}$ by spheres along with a max and min denoted $\infty$ and $-\infty$ respectively. 

\section*{Definitions}
\begin{enumerate}
\item The \textbf{bridge number} of $K$ is the minimal number of maxima required for $h_{K}$ and is denoted $b(K).$ This coincides with the definition of  bridge number as the minimal number of bridges (arcs which are an overpass at least once) in a knot diagram.
\item If $J$ is a knot in $S^{3}$ and $L$ is a knot in an unknotted solid torus $\hat{V}$, the knot $K$ obtained by replacing a tubular neighborhood of $J$ with $(\hat{V}, L)$ is called a \textbf{satellite knot} with \textbf{companion} J and \textbf{pattern} $(\hat{V}, L).$ The least number of times in which a meridian disk of $V$ intersects $L$ is called the \textbf{index} of the pattern.
\item For $K$ a homotopically nontrivial knot in a solid torus $V,$ then $V$ is \textbf{taut} with respect to $b(K)$ if the number of critical points of $h_{T}$ ($T = \partial V$) is minimal subject to the condition that $h_{K}$ has $b(K)$ maxima.  
\item Again if $T = \partial V$ and $\mathcal{F}_{T}$ is the foliation induced by $h_{T},$ then a leaf $\sigma$ corresponding to a saddle singularity consists of two circles $s_{1}$ and $s_{2}.$ If either $s_{1}$ or $s_{2}$ is inessential in $T$, then $\sigma$ is called an \textbf{inessential saddle}. 
\end{enumerate}

\section*{The Pop Over Lemma (Lemma 1)}
This lemma precisely states that if there is an inessential saddle in the foliation $\mathcal{F}_{T},$ after an isotopy there is an inessential saddle $\sigma$ such that $s_{1}$ bounds and disk $D_{1} \subset T$ such that the following hold:
\begin{enumerate}
\item The foliation restricted to $D_{1}$ consists of only disjoint circles and either a max or a min
\item In the level set $L$ containing $\sigma,$ $D_{1}$ cobounds a $3-$ball $B$ with a disk in $\tilde{D}_{1} \subset L - T,$ such that $B$ does not contain $\infty$ or $-\infty$ and $s_{2}$ is completely outside $\tilde{D}_{1}.$
\end{enumerate} The first condition is satisfied by choosing the ``innermost'' inessential saddle in a given level set (if there is only one then it is automatically innermost.) Now $L - \partial D_{1}$ contains two disks $\hat{D}_{1}$ and $\hat{D}_{2}$ which cobound (with $D_{1}$) 3-balls $\hat{B}_{1}$ and $\hat{B}_{3}$ one of which either contain $\infty$ or $-\infty$ and the other contains neither. If $s_{2} \subset \hat{D}_{2}$ then take $B = \hat{B}_{1},$ so suppose $s_{2} \subset \hat{D}_{1}.$ Then take a path $\alpha$ from the max in $D_{1}$ to $\infty$ and a tubular neighborhood of $\alpha$ and ``pop'' $D_{1}$ over $\infty$ using this neighborhood. This puts $\infty$ in $\hat{B}_{1}$ and shrinks $\hat{B}_{2}$ so it doesn't contain $\infty$ and we can take $B$ to be the shrunk version of $\hat{B}_{2}.$  


% useful pictures

\section*{The Pop Out Lemma (Lemma 2)}
This lemma states that if $V$ is taut with respect to $b(K)$ then there are no inessential saddles in $\mathcal{F}_{T}.$ \newline \newline This is proven by contradiction. If $\mathcal{F}_{T}$ contained an inessential saddle we could alter it as in lemma 1. Then we can isotope any part of $K$ contained in the ball down below the disk $\tilde{D}_{1}$ and above any other critical points of $h_{T}$ without changing the number of critical points of $h_{K}.$ Now replacing $T$ by $\tilde{T} = (T - D_{1}) \cup \tilde{D}_{1}$ and tilting  we arrive at a torus containing $K$ such that $h_{K}$ still has $b(K)$ maxima but $h_{\tilde{T}}$ has two fewer critical points. Therefore $V$ was not taut with respect to $b(K)$ and we get our contradiction.


\section*{Lemma 3}
Consider a bicollar (pair of pants) of an essential saddle $\sigma$ in $\mathcal{F}_{T},$ it has three boundary components $c_{1}, c_{2},$ and $c_{3}$ where each $c_{i}$ is parallel to $s_{i}$ and lie in the same level set  for $i = 1, 2$. By an Euler characteristic argument $c_{3}$ must bound a disk containing either a max or a min $m_{\sigma},$ so if there are no inessential saddles then every max or min corresponds to a saddle. \newline \newline
With an essential saddle and the pair of pants as above, since $c_{1}$ and $c_{2}$ are disjoint in the level set $L \approx S^{2},$ they cobound an annulus. If a collar neighborhood of $c_{1} \cup c_{2}$ \textcolor{blue}{in the pair of pants} is contained in $V,$ then $\sigma$ is a \textbf{nested saddle}. 
% insert useful picture here
\newline \newline Lemma 3 states that if $V$ is taut with respect to $b(K),$ then $\mathcal{F}_{T}$ has no nested saddles. Again, this is proved by contradiction. We will assume that $\mathcal{F}_{T}$ contains some nested saddle and then show that this implies $V$ was not taut. First she proved that the highest saddle not be nested assuming that $V$ is knotted. Then you can find ``adjacent'' saddles, one nested and one not nested, where adjacent means there is some cylinder connecting them. Using this cylinder and the disk which one of the components of the upper saddle bounds, construct a  ball not containing $\infty$ or $-\infty$ (this time the torus may intersect the interior) and push the ball below the lower saddle reducing the critical points of the torus. (see pic in Schulten's paper)
% sketch of proof

\section*{Proof of Theorem}
By the preceding lemmas, if $V$ is a knotted solid torus that is taut with respect to $b(K)$ then there are no inessential saddles and no nested saddles. All the saddles $\sigma_{1}, \sigma_{2}, \hdots, \sigma_{n}$ correspond to a max or min $m_{\sigma_{i}}$. Letting $L_{i} = h^{-1}(h(\sigma_{i}))$ be the level sets containing the saddles, then $L_{1} \cup \hdots \cup L_{n}$ cuts $V$ into $3-$balls containing the maxima and minima as well as vertical cylinders. \newline \newline Now consider the satellite knot $K$ with companion $J$ and a pattern $(\hat{V}, L)$ of index $k.$ We can assume $K \subset V = \nu(J)$ and furthermore we can assume that $V$ is taut with respect to $b(K).$ We obtain a morse function on $J$ by collapsing $V$ to it's core, and we get that $b(J) \leq |\text{max of }h_{T}|$ where $T = \partial V.$ Now each max corresponds to a saddle $\sigma,$ and if $D_{1}$ and $D_{2}$ are the disks bounded by $s_{1}$ and $s_{2},$ then $D_{1} \cup D_{2}$ cuts a $3$-ball $B_{\sigma}$ containing $m_{\sigma}$ off of $V.$ Since the index of the pattern is $k,$ at least $k$ strands pass through both $D_{1}$ and $D_{2},$ thus $K$ has at least $k$ maxima in $B_{\sigma}.$ This is true for each maximum of $T,$ so we have that $b(K) \geq k\cdot |\text{max of }h_{T}| \geq k\cdot b(J).$ \newline \newline Now in the special case where $K = K_{1} \# K_{2},$ suppose $b(K_{1}) \geq b(K_{2})$ and in the satellite construction take $K_{1}$ to be the companion knot with $(\hat{V}, K_{2})$ the pattern of index 1. Again we can obtain a Morse function on $K_{1}$ by shrinking $V = \nu(K_{1})$ down to its core. Then for each maximum of $T = \partial V$ there is an associated saddle $\sigma$ and again $s_{1}$ and $s_{2}$ bound disks $D_{1}$ and $D_{2}.$ \textcolor{blue}{If} for each max $|K \cap D_{i}| \geq 2,$ then $K$ must obtain at least two maxima and thus $b(K) \geq 2\cdot b(K_{1}) \geq b(K_{1}) + b(K_{2}) \geq b(K_{1}) + b(K_{2}) - 1.$ \textcolor{blue}{So} we need to consider the case when $|K \cap D_{i}| = 1.$ Recall we can decompose $V$ into $3$-balls containing the minima and maxima of $T$ along with solid vertical cylinders. In a vertical cylinder, a meridinal disk which intersects with $K$ once can be pushed to the bottom of some ``knee'' corresponding to a max. We can then push through the knee leaving a single arc with a single max. How to do this? Let $\alpha$ be the sub arc of $K$ entering the bottom left knee and increasing to the maximum corresponding to this strand. It's safe to assume the second endpoint of $\alpha$ is the highest max of $K$ contained in this knee. Then $\alpha$ along with 3 other arcs bound a disk $\Delta$ in $V$ such that $\Delta \cap K = \alpha$ (see Schulten's pic for why we can do this) and pushing along $\Delta$ we can move all of $K$ except a single arc into the descending cylinder. Now repeat this process until each max of $V$, each vertical cylinder, and all but one min contains a single arc of $K.$ Then $K$ has $b(K_{1})$ maxima (one in each max of $V$) and \textcolor{blue}{at least} $b(K_{2}) - 1$ maxima in the last min of $V.$ Thus $b(K) \geq b(K_{1}) + b(K_{2}) - 1.$
 
\section*{The Whitehead Double}
\textcolor{red}{Have Ryan read this.} In the specific case where the pattern $(\hat{V}, L)$ is the Whitehead double pattern of index 2, we get that for any nontrivial knot $K \subset  S^{3}$ its Whitehead double $K_{WD}$ has bridge number $b(K_{WD}) \geq 2\cdot b(K).$ \textbf{Claim:} $b(K_{WD}) \leq 2\cdot b(K)$ and therefore $b(K_{WD}) = 2 \cdot b(K).$ \begin{proof} Let $K \subset S^{3}$ be a nontrivial knot and $h: S^{3} \rightarrow \mathbb{R}$ a Morse function such that $h_{K}$ has $b(K)$ maxima. Then let $L = K \cup K'$ be the link consisting of $K$ along with a parallel copy of $K, K',$ obtained by the 0-framing of $K.$ This implies that $h_{K'}$ also has $b(K)$ maxima. Choose some minima $m \in K$ and let $m' \in K'$ be the corresponding minima in $K'.$ Now let $\alpha$ be the path (you know the one I'm thinking about) connecting $m$ to $m'$ with no critical points and let $b \approx I \times I$ be a band with core $\alpha$ (i.e. $\alpha = \{0\} \times I$) and with one full twist such that $b \cap L = \{-1, 1\} \times I.$ Then we have (after smoothing corners) that $K_{WD}$ is the result of resolving the band $b,$ i.e. $K_{WD} = L_{b} =( L- (\{-1, 1\} \times I)) \cup (I \times \{-1, 1\}).$ In this construction we have introduced no new maxima, so $K_{WD}$ has $2 \cdot b(K)$ maxima and thus $b(K_{WD}) \leq 2 \cdot b(K).$

\end{proof}

\section*{Various Definitions of Bridge Number}
Schultens defines bridge number as above, the minimal number of maxima required for $h_{K}.$ However, if you Google bridge number the Wikipedia article defines bridge number as the least number of ``bridges'' (any arc that includes at least one overcrossing) in all diagrams of a knot. \textcolor{blue}{We will revisit this definition later.} Another definition which makes sense to me involves the bridge splitting of a knot. \newline \newline A \textbf{$b-$bridge splitting} of a link $L$ inside $S^{3}$ is a decomposition $(S^{3}, L) = (B_{1}^{3}, \alpha_{1}) \cup (B_{2}^{3}, \alpha_{2})$ where each $(B_{i}^{3}, \alpha_{i})$ is a trivial tangle with $b$ arcs. A trivial tangle is one such that all arcs are simultaneously isotopic to the boundary sphere. You can easily isotope a $b-$bridge splitting to a $(b+1)-$bridge splitting, think of this as a stabilization, so you can consider the minimum value $b$ such that $K$ admits a $b-$bridge splitting. This seems to be a reasonable definition for bridge number (https://ldtopology.wordpress.com/2013/02/16/the-bridge-spectrum/), so let's define $\beta(K) := min\{b | K \text{ admits a $b-$bridge splitting}\}.$ \newline \newline \textbf{Claim:} For a knot $K$ in $S^{3},$ the bridge number agrees with this number, $b(K) = \beta(K).$ \begin{proof}Suppose $K \subset S^{3}$ has $b(K) = b.$ Then there exists a ``standard'' Morse function $f: S^{3} \rightarrow \mathbb{R}$ such that $f_{K}$ has $b$ maxima. Assume $K$ is in bridge position with respect to this Morse function, we can raise the maxima and lower the minima if necessary. Suppose $p \in K$ is the highest minima, then there exists an $\epsilon$ such that all maxima occur above $f^{-1}(f(p) + \epsilon)$. Now  $(S^{3}, K) = (f^{-1}([f(p) + \epsilon, \infty], K_{above}) \cup (-\infty, f^{-1}([f(p) + \epsilon], K_{below})$ is a $b-$bridge splitting, thus $\beta(K) \leq b(K)$. \textcolor{red}{THIS IS SKETCHY STARTING NOW} On the other hand, given a $b-$bridge splitting $(S^{3}, K) = (B_{1}^{3}, \alpha_{1}) \cup_{S^{2}_{*}} (B_{2}^{3}, \alpha_{2})$, there is an ambient isotopy of the bridge sphere, $S^{2}_{*}$ to the ``equator'' of $S^{3}.$ In more technical terms, there exists (Jordan-Schoenflies Theorem assuming nice conditions) a smooth map $f: S^{3} \times I \rightarrow S^{3}$ such that 
\begin{enumerate}
\item all $f_{t}: S^{3} \rightarrow S^{3}$ defined by $f_{t}(p) = f(p, t)$ are diffeomorphisms 
\item $f_{0}: S^{3} \rightarrow S^{3}$ is the identity, and 
\item $f_{1}(S^{2}_{*}) = S^{2} \times \{0\} \subset \Sigma S^{2} = S^{3}.$
\end{enumerate}
Then $f(B_{1}^{3}) = (S^{2} \times [0, 1])/ (S^{2} \times \{1\}) \sim *$ and $f(B_{2}^{3}) = (S^{2} \times [-1, 0])/ (S^{2} \times \{-1\}) \sim *$ and the arcs $\alpha_{i}$ are mapped to arcs $f(\alpha_{i})$ with boundary on $S^{2} \times \{0\}.$ After a small isotopy of $f(\alpha_{1})$ we can assume each strand has exactly one maximum with respect to the standard ``height function'' ($h: \Sigma S^{2} \rightarrow \mathbb{R}$ defined by $h(p, t) = \frac{t}{1 - t}$?) on $S^{3}.$ Similarly we can isotope $f(\alpha_{2})$ so each strand has exactly one minima (and no maxima) with respect to this function. Thus we end up with a Morse function with exactly $b$ critical points giving us $b(K) \leq \beta(K).$
\end{proof}  \noindent 
\textcolor{blue}{Later is now. This is the diagrammatic definition of bridge number in Colin Adam's book ``The Knot Book''.} Given a projection of a knot to the plane, an \textbf{overpass} (bridge) is a subarc that goes over at least one crossing and \textcolor{red}{never} goes under a crossing. An overpass is \textbf{maximal} if both of its endpoints occur just before going under a crossing (it couldn't be made any longer.) The bridge number of a projection $P$ is the number of maximal overpasses in the projection, and then the bridge number of a knot $K$ is defined as $b(K) := min\{\text{bridge number of }P : P \text{ is a projection of }K \}.$ \newline \newline \textcolor{blue}{Idea for why this is an equivalent definition:} View $S^{3}$ as the suspension of $S^{2}.$ Now given a planar diagram of a knot, we can embed this diagram onto the equatorial 2-sphere $S_{*}^{2} \subset \Sigma S^{2} = S^{3}.$ For each maximal overpass, perturb up into $S^{3}$ while keeping the endpoints fixed on $S_{*}^{2}.$ The remaining arcs in the diagram will have endpoints just before under crossings and will all be disjoint (using maximality) so we can just consider the isotopy taking them down into $S^{2} \times \{-\epsilon\}.$ Connecting these (smoothly) to either the other ``underpasses'' in this level sphere or to the appropriate overpasses will result into a smooth embedding of the knot which has a natural bridge splitting with the the number of strands equaling the number of maximal overpasses. Now suppose we have a bridge splitting, then there is an isotopy of the bridge sphere onto the equatorial $2-$sphere as discussed above. The arcs in a bridge splitting are simultaneously isotopic to the boundary, so realizing this isotopy will result in a projection of the knot with crossings on $S_{*}^{2}.$ Making sure that every arc in the ball $(S^{2} \times [0, 1])/ (S^{2} \times \{1\}) \sim *$ is an over crossing at every crossing it is involved in will give a knot diagram with the same number of maximal overpasses as there are strands.

\section*{Bridge Number for 2-Knots}
In the Meier-Zupan paper, they define a $(b; c_{1}, c_{2}, c_{3})-$bridge trisection for a knotted surface in $S^{4}.$ Then they call then minimal $b$ such that a surface admits a  $(b; c_{1}, c_{2}, c_{3})-$bridge trisection the bridge number of the surface. They then completely classify all surfaces with bridge number $b \leq 3.$ Then, given a $b-$bridge knot $K$, they come up with a $(3b-2; b, b, b)-trisection$ of the spun knot $S(K).$ In the case $b = 2,$ the spun knot has bridge number equal to 4 because it won't be any of the surfaces with lower bridge number (unknotted 2-sphere, tori, projective planes, and klein bottles.) However a spun $b$-bridge knot will still be isotopic to a knotted $2-$sphere with $b$ maxima. This shows that the MZ definition of bridge number does not equal the minimal number of maxima for a $2-$knot (a spun $2-$bridge knot will have 2 maxima), which is not that big of a deal. Now given a $3-$bridge knot $K$, MZ's $(7, 3)-$bridge trisection could possibly be destabilized so $4 \leq b(S(K)) \leq 7.$ \textcolor{red}{TODO:} Try this for the Perko pair and $8_{10}.$



\end{document}